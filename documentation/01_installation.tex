\section{Installation}
To install the \ac{HIPAcc} framework, which features a Clang-based source-to-source
compiler, a version of Clang is required that was tested with \ac{HIPAcc}.
Therefore, the file {\tt dependencies.sh} lists the revision of Clang (and other
dependencies) the current version of \ac{HIPAcc} works with. In addition to Clang,
also LLVM and libcxx are required. Using Polly is optional.


\subsection{Dependencies}
\begin{itemize}
    \item \href{http://clang.llvm.org}{Clang: a C language family frontend for LLVM}
    \item \href{http://llvm.org}{LLVM compiler infrastructure}
    \item \href{http://libcxx.llvm.org}{libc++: C++ standard library}
    \item \href{http://polly.llvm.org}{Polly: polyhedral optimizations for LLVM} (optional)
    %\item \href{http://www.cloog.org}{CLooG: the chunky loop generator} (optional)
\end{itemize}

Installation of dependencies:
%\lstset{language=bash}
\begin{lstlisting}
cd <source_dir>
git clone http://llvm.org/git/llvm.git
cd llvm && git checkout <llvm_revision>
cd <source_dir>
git clone http://llvm.org/git/libcxx.git
cd libcxx && git checkout <libcxx_revision>
cd <source_dir>/llvm/tools
git clone http://llvm.org/git/clang.git
cd clang && git checkout <clang_revision>
// optional: installation of polly
cd <source_dir>/llvm/tools
git clone http://llvm.org/git/polly.git
cd polly && git checkout <polly_revision>
\end{lstlisting}
Configure and install the software packages using {\tt ./configure} and {\tt
make install}.\\
Note: On GNU/Linux systems, libc++ has to be built using clang/clang++. The
easiest way to do this is to use CMake and to specify the compilers at the command
line: \verb|CXX=clang++ CC=clang cmake ../ -DCMAKE_INSTALL_PREFIX=/opt/local| \\
Note: On Mac OS 10.6, {\tt cxxabi.h} from
\href{http://home.roadrunner.com/~hinnant/libcppabi.zip}{http://home.roadrunner.com/~hinnant/libcppabi.zip}
is required to build libc++ successfully.


\subsection{\ac{HIPAcc} Installation}
Next, you have to download \ac{HIPAcc} from
\href{http://hipacc.sourceforege.net}{http://hipacc.sourceforege.net}.
\ac{HIPAcc} can be downloaded either as versioned tarball, or the latest version
can be retrieved using git. Download the latest release (currently, the tarball
hipacc-0.5.2.tar.gz) or get the latest sources using git:\\
{\tt git clone git://git.code.sf.net/p/hipacc/code hipacc}

To build and install the \ac{HIPAcc} framework, CMake is used. In the main
directory, the file INSTALL contains required instructions:

To configure the project, call cmake in the root directory. A working
installation of Clang/LLVM (and Polly) is required. The llvm-config tool will be
used to determine configuration for \ac{HIPAcc} and must be present in the
environment.

The following variables can be set to tell cmake where to look for certain components:
\begin{itemize}
    \item CMAKE\_INSTALL\_PREFIX:   Installation prefix
    \item CMAKE\_BUILD\_TYPE:       Build type (Debug or Release)
    \item OPENCL\_INC\_PATH:        OpenCL include path\\(e.g., -DOPENCL\_INC\_PATH=/opt/cuda/include)
    \item OPENCL\_LIB\_PATH:        OpenCL library path\\(e.g., -DOPENCL\_LIB\_PATH=/usr/lib64)
    \item CUDA\_BIN\_PATH:          CUDA binary path\\(e.g., -DCUDA\_BIN\_PATH=/opt/cuda/bin)
\end{itemize}

The following options can be enabled or disabled:
\begin{itemize}
    \item USE\_POLLY: Use Polly for kernel analysis (e.g., -DUSE\_POLLY=ON)
    \item USE\_JIT\_ESTIMATE: Use just-in-time compilation of generated kernels
    in order to get resource estimates - option only available for GNU/Linux
    systems
\end{itemize}

A possible configuration may look like in the following:
%\lstset{language=bash}
\begin{lstlisting}
cd <hipacc_root>
mkdir build && cd build
mkdir install_dir
cmake ../ -DCMAKE_INSTALL_PREFIX=./install_dir
make && make install
\end{lstlisting}

